\documentclass[11pt,a4paper]{article}
\usepackage[latin1]{inputenc}
\usepackage{mystuff}
\usepackage{fullpage}
\usepackage{pdfpages}
\usepackage[skins]{tcolorbox}

\usepackage[a4paper, margin=1in]{geometry}
\usepackage{caption}
\captionsetup{font=footnotesize}
\usepackage{algorithm,algpseudocode}

\newcommand{\rad}[1]{%
  \resizebox{5cm}{!}{\mpost{square.#1}}
}

\newcommand{\rads}[1]{%
  \resizebox{3cm}{!}{\mpost{square.#1}}
}

\begin{document}
\title{A linear time algorithm for ranking scores with ties}
\author{Kjell Post \\ Irsta Fotostudio \\ {\tt kjell@irstafoto.se}}
\date{}
\maketitle

\abstract{In competitions, it is useful to present a list
  of winning entries and their scores, but in case of several entries with
  the same score we must present not only their final place, but also if their
  place was tied.
  This memo describes an $O(n)$ algorithm to rank the entries so that each
  entry has a place and a tie marker, e.g., ``2nd place'' or ``Shared 1st place''.}
  
\section{The ranking algorithm}
Consider a competition with a set of scores $S$, assumed sorted:
\begin{tabbing}
  \qquad $S = \{ 0, 0, 10, 75, 76, 76, 76, 77, 77, 77, 78, 79, 80, 80 \}$
\end{tabbing}
When the scores are presented, we would like to display them as follows:

\begin{center}
\begin{tabular}{l r r}
  Shared &  13th place: &   0 \\
  Shared &  13th place: &   0 \\
         &  12th place: &  10 \\
         &  11th place: &  75 \\
  Shared &   8th place: &  76 \\
  Shared &   8th place: &  76 \\
  Shared &   8th place: &  76 \\
  Shared &   5th place: &  77 \\
  Shared &   5th place: &  77 \\
  Shared &   5th place: &  77 \\
         &   4th place: &  78 \\
         &   3th place: &  79 \\
  Shared &   1st place: &  80 \\
  Shared &   1st place: &  80
\end{tabular}
\end{center}

\noindent
We assume $S$ can be viewed both as a set and as a list
$S[0] \ldots S[N-1]$ where $N$ is the number of scores.
The following algoritm annotates each score with
attributes \emph{place} (denoting the score's final place)
and \emph{tied} (a boolean, denoting if the place is tied).

\begin{tabbing}
\qquad  \textbf{procedure} \emph{rank}$(S)$: \\
\qquad  \qquad \= $N \leftarrow \emph{length}(S)$ \\
\qquad         \> \textbf{if} $N = 0$ \textbf{return} \\
\qquad         \> $p \leftarrow 1$ \\
\qquad         \> $\emph{topScore} \leftarrow \infty$ \\
\qquad         \> \textbf{for} $k \in \{ N-1, N-2, \ldots 1, 0 \}$ \= \emph{// Visit each score in ascending order} \\
\qquad  \qquad\qquad \= \textbf{if} $S[k] < \emph{topScore}$ \> \emph{// A new distinct score $\Rightarrow$ a new place} \\ 
\qquad  \qquad\qquad\qquad \= $p \leftarrow N - k$ \\
\qquad                     \> $\emph{topScore} \leftarrow S[k]$ \\
\qquad  \qquad\qquad \= $S[k].\emph{place} \leftarrow p$ \\
\qquad               \> $S[k].\emph{tied} \leftarrow \emph{false}$ \\
\qquad               \> \textbf{if} $k < N-1$ \textbf{and} $S[k+1].\emph{place} = p$ \emph{// If previous score is the same, we have a tie}\\
\qquad  \qquad\qquad\qquad \= $S[k].tied \leftarrow \emph{true}$ \\
\qquad                     \> $S[k+1].tied \leftarrow \emph{true}$ 
\end{tabbing}

\section{Python implementation}
\begin{verbatim}
#!/usr/bin/env python

photos = [
     { 'score':  0 },
     { 'score':  0 },
     { 'score': 75 },
     { 'score': 76 },
     { 'score': 76 },
     { 'score': 76 },
     { 'score': 77 },
     { 'score': 77 },
     { 'score': 77 },
     { 'score': 78 },
     { 'score': 79 },
     { 'score': 80 },
     { 'score': 80 },
     { 'score': 10 },
]

# sort photos by score
photos.sort(lambda x,y : cmp(x['score'], y['score']))

def rank(s):
    N = len(s)
    if N == 0:
        return
    p = 1
    topScore = 999999
    # visit s[N-1], s[N-2], ... s[1], s[0]
    for k in range(N-1, -1, -1):
        if s[k]['score'] < topScore:
            p = N - k
            topScore = s[k]['score']
        s[k]['place'] = p
        s[k]['tied'] = False
        if k < N-1 and s[k+1]['place'] == p:
            s[k]['tied'] = True
            s[k+1]['tied'] = True
    
rank(photos);
for p in photos:
    if p['tied']:
        print 'Shared %4s place: %3s' % (p['place'], p['score'])
    else:
        print '       %4s place: %3s' % (p['place'], p['score'])
\end{verbatim}

\end{document}
